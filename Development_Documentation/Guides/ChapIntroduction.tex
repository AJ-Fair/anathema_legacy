\chapter{Introduction}
\section{About Anathema}
Anathema is a computer toolkit for players of the role playing game of ``Exalted'', published by White Wolf Game Studious of Atlanta, Georgia. It is written in Java and published under the General Public License (GPL).

\section{About this Guide}
This document aims to provide users with an overview of the various functions of Anathema. All modes of usage will be explored and explained in-depth. For those interested in extending the program, instructions will be given. An additional section explains the basic happenings inside the program to developers.
Each chapter assumes that you have the base knowledge required for using the functions provided. This is no guide to Exalted character design (there are far more proficient people working on that), neither a guide to XML, nor to Java programming. Each of these issues goes far beyond the scope of the guide, but pointers will be given where required.

\section{A History of Exalted Toolkits}
In the beginning, there was White Wolf. The Limited Edition of Exalted featured the demo of a ``soon to come'' character generator for their brand new game. Years passed, Exalted was a huge success, but of the character generator, we never heard again.

Next came Jake Baker's Unofficial Exalted Character Generator \cite{UECG}. It support a wide variety of character types but was unfortunately not continued into 2003.

That year, however, gave birth to the first public version of EdExalted \cite{EdExalted}. Edward A. Ostrowski provided us with a set of toolkits that is to this date unrivaled in the sheer mass of features. It supports virtually every type of character published for Exalted, comes with a huge database of items and charms and is known to be quite customizable.

The mere existence of EdExalted begs the question: Why Anathema?

\section{Why Anathema?}
\begin{description}
\item[Usability.]As great a program EdExalted is when it comes to features, we were quite sad about the user interface. 
Another issue we had were the character sheets generated by the program. While they teem with information, we found them lacking in layout.

\item[Management.]EdExalted is clearly geared towards managing characters, and it does a great job at that. Nevertheless, we realised that there are other issues one faces over the course of a campaign, and we longed for a program to support us with these.

\item[Finally: Curiosity.] Exploring the challenges of modelling a complex set of rules, we wished to see for ourselves what was possible and what was not.
\end{description}