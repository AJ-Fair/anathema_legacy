\chapter{Customizing Natures}
This chapter details the process of customizing the "willpower recovery" conditions for 
exisiting natures and describes a way of adding completely new natures. 

The program comes complete with all natures from the Exalted core rulebook, but the willpower recovery conditions are omitted -- you guessed it -- for copyright reasons. 

\section{Concepts}
A nature in Anathema is represented by three parts: An unique ID for identification purposes,
the nature's name as it appears in the program and on character sheets and a
willpower recovery condition. 

For the core book's natures, ID and Name are both set to the nature's name (as given in the core rulebook, including capitalization), while the condition only states the page containing the description.

\section{Adding Willpower Recovery Conditions}
To make more use of the "`Willpower"'-part of the nature, you may want to replace the stub descriptions by more useful ones.
To do so, create a plain text file called "custom.properties" in the Anathema main directory.

Within, you can specify new conditions in a 'key=value' format. The first part, 'key', is a technical looking expression the program looks up and replaces with the second part, the far more legible 'value'. Keys for the willpower recovery conditions adhere to the format \texttt{Nature.[Insert ID].Text}. The Value-part is whatever you want to make of it.

\paragraph{Note:} "Gain Willpower" is always included in front of the text, so you need to enter the conditional
phrase ("when...", "any time...") only.

Once you've entered all the conditions you need, save the file and restart Anathema. They'll automatically be used for all existing and newly created characters.

\section{Creating New Natures}
The first step to creating new natures is to create a new file "natures.xml" in the subdirectory "data", then editing the content of this file. The nature file must begin with a root element \texttt{<naturelist>} and then specify any number of
nature entries, before ending with a final \texttt{</naturelist>}.

Each nature entry consists of a single line announcing the new nature's ID to the program, and looks like this:
\texttt{<nature id="[Insert ID]" />}

This file is read during the Anathema launch process and it's contents are added to the
nature list. 

\paragraph{Example:} A very simple natures.xml file could look like this:\\
\texttt{<naturelist>\\
	<nature id="Customizer" />\\
</naturelist>}


However, newly created natures show as "\#\#Nature.[ID].Name\#\#" in the character concept screen. You will have to tell the program what to put in its place, and attach a willpower recovery condition.

To this end, re-open the "custom.properties" file you created when text for the pre-defined natures, and add the recovery conditions as before. Additionally, you will want to tell the program what the new nature is called - just add another 'key=value' line, with \texttt{Nature.[Insert ID].Name} as the key and the name as its value.

\paragraph{Example:} For our example nature, these lines could read:\\
\texttt{Nature.Customizer.Name=Customizer\\
Nature.Customizer.Text=using variations of pre-defined patterns.}