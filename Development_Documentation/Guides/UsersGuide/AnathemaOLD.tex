\section{Character}

\subsection{Creating a new character}
Together, we will walk through the creation of a new character, from initial creation up to spending the first experience points. Afterwards, everything is up to you.

For purposes of this guide, a ``Loyal Abyssal'' seems like a good choice, since they have some options others are lacking. Confirm the choices made by clicking ``OK''.

After a moment's notice, an item will open.
Three separate areas can be distinguished: The tab area on the upper left, granting access to the different aspects of a character, the overview on the lower left, at all times displaying information about points spent, and the main view, in which the options pertaining to the tab selected are shown.

Right now, we are on the ``Description'' panel, with fields identical to the ``Character Description'' item outlined above (\ref{sec:CharacterDescription}).

Adjusting Values
Throughout Character Creation, you will notice sets of light grey dots - these are to adjust your characters trait values. To select the number of dots assigned to a certain trait, just click on the dots and they will be filled in - alternatively click and keep the mousebutton pressed to be able to drag the value until it's set to the amount you want it to be.
If you want to remove dots, click on the previous dot to remove the next. To remove the first dot, click just before it or in the first third of the dot - again, dragging the selection and releasing the mouse after the value is adjusted will make things easier.
Caste \& Favored Abilities
On the abilities tab, you will note a button in front of each ability. Clicking this button will add it to your Favored Abilities, marked by a highlight.
Caste Abilities are marked with the caste mark.
Magic
The Magic section allows you to choose Charms, spells, and create combos, each on a separate page. Except for some Martial Arts, only Solar Charms can be currently selected. Use the drop list to choose the Group (Ability the Charm is in on and the complete cascade for it will appear. Select Charms by clicking on them and they should turn yellow, indicating a learned charm.This cascade works the same as the separate Cascade function described earlier.
If a Charm is grayed out that means the character has not yet met all prerequisites.
Spells can be learned by selecting them from in the left listbox and clicking the green arrow to the right. To unlearn a spell, select it, then click the red "X".
Combo creation is similar to spell selection: Simply select one charm at a time and click the green arrows to add it to the current combo, afterwards enter a name and short description of the combo, then click the green + to finalize the combo. The center button can be used to clear the combo currently edited and start over.
Overview
The Overview on the left will keep track of the choices you made, automatically adjusting if you change things. Colors and font show the state of the item displayed:
Purple indicates that you still need to spend points, while black means that everything allotted is spend. Items on which you spend bonus points (which are calculated automatically) are shown bold. If the Bonus Points counter at the bottom becomes red, this means that you have overextended your amount allowed bonus points.
Number in parentheses indicated that the characters absolute maximum has somehow been raised above the maximum (eg. a larger than usual Essence Pool), while a second number delimited by a "+" notifies you of additional Points to spend.
For example, if I put 8 dots in Physical, and 6 in Mental, then remove 2 dots from Physical, the Promary will list 6/8 meaning only 6 points of 8 have been spent. If I then add them to Mental, it becomes the Primary and is then listed as 8/8. If I spend a 9th dot the Primary line becomes Bold, and the Bonus Points, at the bottom of the left side, is incremented to show that 4 points has been spent to increase the Attribute.

\subsection{Experience}
Once character creation is finished, the character can be further advanced within the program, but first, you have to convert it to 'experienced mode' by clicking the equinymous entry in the 'Character' menu.
After clicking, you'll notice a change in the overview window - it now shows your current XP breakdown - as well as a new tab for the character - within, you'll find a table in which to enter the experience awards for your character.To do so, just add an entry by clicking the '+' and adjust the number. You can also add a comment for later reference.
From now on, your character's stats cannot be lowered below the values you've set during creation, and any adjustments you make will be paid for with experience, their cost being calculated automatically.